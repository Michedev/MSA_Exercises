\documentclass[10pt,a4paper]{article}
\usepackage[latin1]{inputenc}
\usepackage{amsmath}
\usepackage{amsfonts}
\usepackage{amssymb}
\usepackage{graphicx}
\usepackage[\shortlabels]{enumerate}
\begin{document}
	\author{Michele De Vita}
	\title{Solution of week 2}
	\maketitle
	\noindent
	\begin{enumerate}
		\item \textit{drop if !inlist(schoolid, 1224, 1288, 1296, 1308, 1317)} \\
		\textit{keep mathach minority schoolid ses}
		\item \begin{itemize}
			\item cmd for continuos variables: \textit{tabstat  ses mathach, stat(mean sd)} \\
			\item cmd for categorical: \textit{table minority}
			\item continuos vears by schoolid: \textit{tabstat ses mathach, stat(mean sd) by(schoolid)}

		\end{itemize}
			\item \textit{histogram mathach} \\
			\textit{graph box mathach}
			\item \textit{scatter mathach ses \\
				twoway scatter mathach ses, by(schoolid)
			}
		\item \textit{regress mathach ses}\\
		Parameters interpretation:\\

		$ b) $ With intercept  equal to 11.45652 we can say that if we have $ ses = 0 $ we have a math achievement equal to 11.45652\\
		The coefficient $ \beta_1 $ before $ ses $ say that if we increment ses (socio economic status) by one we have an increment of mathach by 3.306963\\
		The residual $ \epsilon_i  ~ N(0, \sigma^2)$ with $ \sigma = 6.4708 \text{ and } \sigma^2=41.8712$ tell us about the average deviation from the regression line.\\ \\
		$ c) $ Since the p-value is less that $ 0.05 $ we reject the null hypothesis $ H_0 = \beta_2 = 0 $ 

	\item \textit{predict yhat}
	
	\item \textit{twoway (scatter mathach ses) (lfit yhat ses) \\
		twoway (scatter mathach ses) (lfit yhat ses), by(schoolid) \\
		twoway (scatter mathach ses) (lfit yhat ses) (lfit mathach ses), by(schoolid)
	} \\
	Comment about differences between  mean mathach, ses and yhat, ses:
	For the school 1224 and 1288 the lines respectively $ \hat{y} \text{ and } $ are very similar, while the other three are a little bit different
	\item 
	\begin{enumerate}[a)]
		\item \textit{tabulate schoolid, generate($ schoolid\_ $)}
		\item \textit{regress mathach schoolid$\_$2 schoolid$\_$3 schoolid$\_$4 schoolid$\_$5}
		The coefficient before the schoolids dummy variables say the differences between the school with $ schoolid = 1224 $ and the others. The coefficients say that:
		\begin{itemize}
			\item $ \beta_1 =  3.795353  $ say that the school with $ schoolid = 1228  $ has an average mathach superior by 3.795353 from the school with $ schoolid = 1224 $.
			\item $ \beta_2 =  -2.079489  $ say that  the school with $ schoolid =  1296 $ has a mean mathach lower by 2.079489 respect to school with $ schoolid = 1224 $
			\item .. and so on with the other two coefficient and respectively the other two id of the schools
		\end{itemize}
		\item \textit{testparm schoolid$\_* $} We reject the null hypothesis of all dummy variables are 0 because the p-value is 0.0015, so less than 0.025
	\end{enumerate}
	\item \textit{regress mathach c.ses$ \#\# $i.schoolid$ \_* $}
	\end{enumerate}


	

	

\end{document}